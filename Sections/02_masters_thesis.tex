\sectionframe{Results of my Masters Thesis}
\section{Masters Thesis}

\begin{frame}{Archetypal Model}
	\begin{figure}
		\stackunder[5pt]{
			\includegraphics[width=0.4 \textwidth]{Figs/og_model_period.png}
		}{Original model}
		\qquad
		\stackunder[5pt]{
			\includegraphics[width=0.41 \textwidth]{Figs/archetypal_model_period.png}
		}{Archetypal model}
	\end{figure}
\end{frame}

\begin{frame}{Archetypal Model}
	\begin{figure}
		\centering
		\stackunder[5pt]{
			\includegraphics[width=0.3 \textwidth]{Figs/archetypal_model_cycle_c16.png}
		}{$C_{16}:\:\Cycle{\A^6\B^2\C^6\D^2}$}
		\stackunder[5pt]{
			\includegraphics[width=0.3 \textwidth]{Figs/archetypal_model_cycle_d16.png}
		}{$D_{16}:\:\Cycle{\A^6\B^2\C^5\D^3},\Cycle{\A^5\B^3\C^6\D^2}$}
		\stackunder[5pt]{
			\includegraphics[width=0.3 \textwidth]{Figs/archetypal_model_cycle_e16.png}
		}{$E_{16}:\:\Cycle{\A^5\B^3\C^5\D^3}$}
	\end{figure}
\end{frame}

\begin{frame}{Archetypal Model Equations}
	\vspace{-1.0em}
	\begin{align*}
		x_{n+1} = f(x_n) \mod 1
	\end{align*}
	\begin{align*}
		f(x) & = \begin{cases}
			         g(x)                                        & \text{ if } x < \frac{1}{2} \\
			         g\left(x - \frac{1}{2}\right) + \frac{1}{2} & \text{ else}
		         \end{cases}
	\end{align*}
	\begin{align*}
		g(x) & = \begin{cases}
			         g_L(x) = a_L \cdot x^2 + b_L \cdot x + c_L & \text{ if } x < \frac{1}{4} \\
			         g_R(x) = b_R \cdot x + c_R                 & \text{ else}
		         \end{cases}
	\end{align*}
\end{frame}

\begin{frame}{Archetypal Model Equations}
	\vspace{-1em}
	\begin{columns}
		\begin{column}{.7 \textwidth}
			Fixed parameters:
			\begin{align*}
				a_L = 4 \text{ and } b_L = -\tfrac{1}{2}
			\end{align*}
			Variable parameters
			\begin{align*}
				 & c_L, b_R, c_R                                                                                                              \\
				\text{where} \qquad
				 & c_L = \beta,                                                                                                               \\
				 & b_R = -4 \cdot g_R\left(\tfrac{1}{4}\right) + 4 \cdot g_R\left(\tfrac{1}{2}\right),                                        \\
				 & c_R = 2 \cdot g_R\left(\tfrac{1}{4}\right) - 1 \cdot g_R\left(\tfrac{1}{2}\right),                                         \\[1em]
				\text{and} \qquad
				 & g_R\left(\tfrac{1}{4}\right) = \alpha, \text{and } g_R\left(\tfrac{1}{2}\right) = \tfrac{1}{2} + \epsilon \text{ is fixed}
			\end{align*}
		\end{column}
		\begin{column}{.3 \textwidth}
			\begin{figure}
				\centering
				\includegraphics[height=.5 \textheight]{Figs/archetypal_model_parameter_effects_illustration.png}
			\end{figure}
		\end{column}
	\end{columns}
\end{frame}

\begin{frame}{Period-Adding?}
	\vspace{-1em}
	\begin{itemize}
		\item Chains overlap
		\item What happens, when they no longer overlap?
		      \vspace{1em}
		\item Prediction: Period-Adding
		      %\item Turns up often
		      %\item E.g. Edge of main bubble of mandelbrot set
	\end{itemize}
	\begin{figure}
		\includegraphics[height=.5 \textheight]{Figs/Trees/ClassicalAdding/adding.png}
	\end{figure}
\end{frame}

\begin{frame}{Period-Adding?}
	\begin{figure}
		\includegraphics[width=.4 \textwidth]{Figs/archetypal_model_regions_drifting_apart.png}
		\quad
		\includegraphics[width=.4 \textwidth]{Figs/archetypal_model_adding_like_period_corner.png}
	\end{figure}
\end{frame}

\begin{frame}{Period-Adding?}
	\vspace{-1em}
	\begin{figure}
		\includegraphics[width=.4 \textwidth]{Figs/archetypal_model_full_add_hor_2D.png}
		\qquad
		\includegraphics[width=.4 \textwidth]{Figs/archetypal_model_full_add_hor_1D.png}
	\end{figure}
\end{frame}

\begin{frame}{Period-Adding?}
	\vspace{-1em}
	\begin{figure}
		\includegraphics[width=.7 \textwidth]{Figs/Trees/FullArchetypal/adding.png}
	\end{figure}
\end{frame}

%%% Local Variables:
%%% mode: latex
%%% TeX-master: "../Vortrag_Frauenhofer_Weik"
%%% End:
